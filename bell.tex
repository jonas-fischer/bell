\documentclass[twoside,colorback,accentcolor=tud4c,11pt]{tudreport}

\usepackage{ngerman}
\usepackage[utf8]{inputenc} 
\usepackage[T1]{fontenc}
\usepackage{cancel}
\usepackage{mathtools}
\usepackage{float}
\usepackage{hyperref}
\usepackage{braket}

\title{VERSUCH 4.2: Test einer Bell'schen Ungleichung}

\subtitle{
\begin{tabular}{p{4cm}ll} 
 Name & Dominik Pfeiffer   &   Jonas Fischer\\
 Matrikelnummer & 2913632  & 2240758 \\
 E-mail& \textaccent{dominik@diepfeiffers.de} & \textaccent{jonas.fischer.42@gmail.com}\\
 \\Versuchsbetreuung & Thorsten Führer \\
 Durchführung& 28.11.2016 \\
 Abgabetermin& 19.12.2016
 \end{tabular}}
\institution{•}
\sponsor{Wir erklären hiermit, dass die vorliegende Arbeit eigenständig, ohne fremde Hilfe und mit der angegebenen Literatur erarbeitet wurde. Alle Passagen aus Literatur und Internet sind als solche gekennzeichnet. Diese Arbeit liegt weder in gleicher noch ähnlicher Weise einer Prüfungskommission vor.\\\\ 
\begin{tabular}{lp{2em}lp{2em}l}
 \hspace{4cm}   && \hspace{4cm}  && \hspace{4cm}
 \\\cline{1-1}\cline{3-3}\cline{5-5}
    Darmstadt, den \today && Dominik Pfeiffer && Jonas Fischer 
\end{tabular}  
 }   
\begin{document}

\maketitle 

\tableofcontents

\chapter{Ziel des Versuchs}
Dieser Versuch behandelt das von John S. Bell zu einem Experiment formulierte Einstein-Podolsky-Rosen Paradox (EPR Paradox). Albert Einstein, Boris Podolsky und Nathan Rosen versuchten durch ein Paradox, welches sowohl die spezielle Relativitätstheorie als auch die Heisenbergsche Unschärferelation zu verletzten schien, die Grenzen der Quantenmechanik aufzuzeigen.
Wird nun im Verlaufe des Versuches die aus dem Paradox resultierende Bellsche Ungleichung verletzt, so spricht dies für eine korrekte Beschreibung der Vorgänge durch die Quantenmechanik und dafür, dass sie nicht durch eine klassische Theorie mit versteckten Variablen ersetzt werden kann.
\chapter{Physikalische Grundlagen}
\section{Verschränkung}
Als Verschränkung bezeichnet man ein Phänomen der Quantenmechanik, welches beschreibt, dass sich bei einer Messung an einem Subsystem eines Mehr-Teilchen Systemes das Ergebnis eines anderen Subsystemes entscheidend ändert und so keine Rückschlüsse auf das Gesamtsystem mehr gezogen werden können. Der Begriff der Verschränkung wurde 1935 von Erwin Schrödinger geprägt. Mathematisch ausgedrückt bedeutet Verschränkung, dass das System nicht mehr durch die Summe individueller Wellengleichungen der Subsysteme beschrieben wird, sondern über eine überlagerte Wellenfunktion.
Wir betrachten folgendes Beispiel polarisationsverschränkter Photonen:
$\Ket{H}$ und $\Ket{V}$ stellen die Polarisationsrichtungen Vertikal und Horizontal dar. Aus diesen beiden ergibt sich für den gekoppelten Raum die beiden Basisvektoren $\Ket{H,V}$ und $\Ket{V,H}$ und darüber der verschränkte Zustand:
\begin{equation}
\Ket{\Psi}=\frac{1}{\sqrt{2}}\cdot\left(\Ket{H,V}\pm\Ket{V,H}\right)
\end{equation}
Die Erzeugung solcher Photonenpaare wird im nächsten Absatz näher diskutiert. In diesem Versuch erzeugt der Aufbau und Konfiguration immer Photonenpaare, sodass beide Photonen entweder horizontal $\Ket{H,H}$ oder vertikal $\Ket{V,V}$ polarisiert sind. Die Wellenfunktion dieser Verschränkten Photonen hat folgende Form:
\begin{equation}
\Ket{\Psi}_{Bell}=\frac{1}{\sqrt{2}}\cdot\left(\Ket{H,H}\,+\,\Ket{V,V}\right)
\end{equation}
Die Verschränkung der beiden Photonen hat zur folge, dass sobald die Polarisation eines der beiden durch eine Messung bestimmt wird, sofort der Polarisationszustand des zweiten Photons ebenfalls festgelegt ist.
Diese instantane Änderung der Zustände in Verschränkten Systemen störte A. Einstein, N. Rosen und B. Podolsky, sodass sie durch ihr Paradoxon diesen Umstand zu erklären versuchten.
\subsection{Erzeugung polarisationsverschränkter Zustände}
Erzeugung polrisationsverschränkter Photonen!
\section{Das Einstein-Podolsky-Rosen Paradoxon}
A. Einstein sprach zunächst von der 'spooky action at distance' (spukhafte Fernwirkung). 
\section{Bell'sche Ungleichung}
Jonas
\section{Vergleich Quantenmechanik und klassische Theorie}
Jonas?
\section{Halbleiter}
\chapter{Versuchsdurchführung und Auswertung}
\section{Vorbereitende Messungen}
•
\section{Messungen}
\subsection{Aufbau}
\subsection{Erste Messung}
\subsection{Zweite Messung}
\chapter{Fazit}	
•
\renewcommand{\bibname}{Literatur}
\begin{thebibliography}{0}
\bibitem {refname} •

\end{thebibliography}
\end{document}    
