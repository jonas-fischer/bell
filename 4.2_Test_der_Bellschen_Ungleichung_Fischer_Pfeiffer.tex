\documentclass[twoside,colorback,accentcolor=tud4c,11pt]{tudreport}

\usepackage{ngerman}
\usepackage[utf8]{inputenc} 
\usepackage[T1]{fontenc}
\usepackage{cancel}
\usepackage{mathtools}
\usepackage{float}
\usepackage{hyperref}
\usepackage{braket}
\usepackage{amsmath}
\usepackage{units}
\usepackage{siunitx}
\usepackage{graphicx}
\usepackage{subfig}
\usepackage{gensymb}

\title{VERSUCH 4.2: Test einer Bell'schen Ungleichung}

\subtitle{
\begin{tabular}{p{4cm}ll} 
 Name & Dominik Pfeiffer   &   Jonas Fischer\\
 Matrikelnummer & 2913632  & 2240758 \\
 E-mail& \textaccent{dominik@diepfeiffers.de} & \textaccent{jonas.fischer.42@gmail.com}\\
 \\Versuchsbetreuung & Thorsten Führer \\
 Durchführung& 28.11.2016 \\
 Abgabetermin& 19.12.2016
 \end{tabular}}
\institution{Institut für Angewandte Physik}
\sponsor{Wir erklären hiermit, dass die vorliegende Arbeit eigenständig, ohne fremde Hilfe und mit der angegebenen Literatur erarbeitet wurde. Alle Passagen aus Literatur und Internet sind als solche gekennzeichnet. Diese Arbeit liegt weder in gleicher noch ähnlicher Weise einer Prüfungskommission vor.\\\\ 
\begin{tabular}{lp{2em}lp{2em}l}
 \hspace{4cm}   && \hspace{4cm}  && \hspace{4cm}
 \\\cline{1-1}\cline{3-3}\cline{5-5}
    Darmstadt, den \today && Dominik Pfeiffer && Jonas Fischer 
\end{tabular}  
 }   
\begin{document}

\maketitle 

\tableofcontents

\chapter{Ziel des Versuchs}
Dieser Versuch behandelt das von John S. Bell zu einem Experiment formulierte Einstein-Podolsky-Rosen Paradox (EPR Paradox). Albert Einstein, Boris Podolsky und Nathan Rosen versuchten durch ein Paradox, welches sowohl die spezielle Relativitätstheorie als auch die Heisenbergsche Unschärferelation zu verletzten schien, die Grenzen der Quantenmechanik aufzuzeigen.
Wird nun im Verlaufe des Versuches die aus dem Paradox resultierende Bellsche Ungleichung verletzt, so spricht dies für eine korrekte Beschreibung der Vorgänge durch die Quantenmechanik und dafür, dass sie nicht durch eine klassische Theorie mit versteckten Variablen ersetzt werden kann.
\chapter{Physikalische Grundlagen}
\section{Verschränkung}
Als Verschränkung bezeichnet man ein Phänomen der Quantenmechanik, welches beschreibt, dass sich bei einer Messung an einem Subsystem eines Mehr-Teilchen Systemes das Ergebnis eines anderen Subsystemes entscheidend ändert und so keine Rückschlüsse auf das Gesamtsystem mehr gezogen werden können. Der Begriff der Verschränkung wurde 1935 von Erwin Schrödinger geprägt. Mathematisch ausgedrückt bedeutet Verschränkung, dass das System nicht mehr durch die Summe individueller Wellengleichungen der Subsysteme beschrieben wird, sondern über eine überlagerte Wellenfunktion.
Wir betrachten folgendes Beispiel polarisationsverschränkter Photonen:
$\Ket{H}$ und $\Ket{V}$ stellen die Basis der Polarisationsrichtungen Vertikal und Horizontal dar. Die Besonderheit verschränkter Zustände ist nun, dass sie zwar Elemente des Tensorproduktraumes $\Ket{V}\otimes\Ket{H}$ sind, jedoch nicht über das Tensorprodukt der Elemente der Teilräume dargestellt werden können.
Die Erzeugung solcher Photonenpaare wird im nächsten Absatz näher diskutiert. In diesem Versuch erzeugt der Aufbau und Konfiguration immer Photonenpaare, sodass beide Photonen entweder horizontal $\Ket{H,H}$ oder vertikal $\Ket{V,V}$ polarisiert sind. Der Zustand dieser Verschränkten Photonen hat folgende Form:
\begin{equation}\label{bellzustand}
\Ket{\Psi}_{Bell}=\frac{1}{\sqrt{2}}\cdot\left(\Ket{H,H}\,+\,\Ket{V,V}\right)
\end{equation}
Schreibt man diesen Zustand als Tensorprodukt, so ergibt sich:
\begin{equation}
\Ket{\Psi}_{Bell}=\frac{1}{\sqrt{2}}\left(1,0,0,1\right)^{T}
\end{equation}
Diesen Zustand kann man nicht als Tensorprodukt zweier Elemente $\left(a,b\right)\,und\,\left(c,d\right)$ der Teilräume schreiben:
\begin{equation}
\left(\begin{array}{c}a\\b\end{array}\right)\otimes\left(\begin{array}{c}c\\d\end{array}\right)=\left(\begin{array}{c}ac\\ad\\bc\\bd\end{array}\right)\neq\left(\begin{array}{c}1\\0\\0\\1\end{array}\right)
\end{equation}
Die Verschränkung der beiden Photonen hat zur folge, dass sobald die Polarisation eines der beiden durch eine Messung bestimmt wird, sofort der Polarisationszustand des zweiten Photons ebenfalls festgelegt ist.
Diese instantane Änderung der Zustände in Verschränkten Systemen störte A. Einstein, N. Rosen und B. Podolsky, sodass sie durch ihr Paradoxon die vermeintlichen Fehler der Quantenmechanik aufzuzeigen.
\subsection{Erzeugung polarisationsverschränkter Zustände}\label{epz}
Für diesen Versuch werden polarisationsverschränkte Photonen benötigt, um die Bellsche Ungleichung und damit die Quantenmechanik zu testen. Diese Photonenpaare sollen wie bereits oben beschrieben im Bellzustand $\Ket{\Psi}_{Bell}=\frac{1}{\sqrt{2}}\cdot\left(\Ket{H,H}\,+\,\Ket{V,V}\right)$ sein. Der Prozess, in dem die Photonenpaare erzeugt werde heißt 'spontaniuos parametric downconversion' (SPDC oder kurz DC). Er findet in einem Barium-Borat Kristall (BBO) statt und stellt den zeitumgekehrten Prozess der Frequenzverdopplung dar.
Damit der Prozess ablaufen kann muss der Kristall einige Anforderungen erfüllen:
\begin{itemize}
\item Der Kristall muss doppelbrechend sein, also für verschiedene Polarisationen des Lichtes verschiedene Brechungsindices und damit verschiedene Ausbreitungsgeschwindigkeiten haben
\item Um bei der Umwandlung des Pumpphotons in die beiden Signal und Idler Photonen die Impulserhaltung zu gewährleisten muss der Kristall nichtlineare Phasenangleichung (nonlinear phasematching) erfüllen. Dafür muss der Kristall so präpariert werden, dass die einlaufenden Pumpphotonen $\omega_{Pump}$ und die auslaufenden Signal und Idler Phototonen $\omega_{signal}=\omega_{idler}=\frac{\omega_{Pump}}{2}$ den gleichen Brechungindex sehen.
\end{itemize}
Im folgenden wird aufgrund der Komplexität der SPDC der zeitumgekehrte Prozess, also die Frequenzverdopplung kurz erklärt:
Ein bekanntes Anwendungsbeispiel für Frequenzverdopplung sind grüne Laser(-pointer). Zwei Infrarotphotonen niedrigerer Energie werden in einen frequenzverdoppelnden Kristall eingestrahlt und werden dort zu einem Photon höherer Energie konvertiert.
Licht als hochfrequente Elektromagnetische Welle (EM-Welle) hat die Eigenschaft, dass es beim durchqueren von Materie zu einer Polarisation dieser kommen kann, ausgelöst durch den elektrischen Feldanteil und eine Ladungsverschiebung innerhalb der Materie. Stellt man nun den Zusammenhang zwischen E-Feld und der Polarisation her findet man:
\begin{equation}
P=\varepsilon_{0}\left(\chi_{1}E+\chi_{2}E^{2}+\chi_{3}E^{3}+...\right)
\end{equation}
Wird nun Licht zweier Frequenzen $\omega{1,2}$ in den Kristall eingestrahlt, so ergibt sich für die Polarisation zweiter Ordnung mit folgender Beschreibung der einlaufenden Welle:
\begin{equation}
E\left(t\right)=E_{1}\cdot\,e^{-i\omega_{1}t}+E_{2}\cdot\,e^{-i\omega_{2}t}\,+c.c.
\end{equation}
Dabei beschreibt c.c. Den komplex-konjugierten Teil der Welle.
\begin{equation}
P^{2}(t)=\chi_{2}E^{2}(t)
=\chi_{2}\left[E_{1}^{2}(t)e^{-i2\omega_{1}t}+E_{2}^{2}(t)e^{-i2\omega_{2}t}+2E_{1}E_{2}e^{-i(\omega_{1}+\omega_{2})t}+2E_{1}E_{2}^{*}e^{-i(\omega_{1}-\omega_{2})t}+c.c.\right]+2\chi_{2}\left[E_{1}E_{1}^{*}+E_{2}E_{2}^{*}\right]
\end{equation}
Auch für diese Prozess muss der Kristall so präpariert werden, dass sich die beiden Einlaufenden Wellen phasenrichtig überlagern. Den Part mit $\omega_{1}+\omega_{2}$ bezeichnet man hierbei als Summenfrequenz.\\
\\In diesem Experiment wird ein Barium-Borat-Kristall (BBO) verwendet, der SPDC-Typ I fähig ist. Als Typ I bezeichnet man die SPDC-Kristalle, bei denen die entstehenden Photonen mit gleicher Polarisation auf einem Konus ausgestrahlt werden, Typ II hingegen erzeugt orthogonal zueinander polarisierte Photonen.
Der Kristall in diesem Experiment besteht aus zwei BBO-Kristallen, die 90° zueinander verkippt sind, um die Verschränkung der Photonen zu erreichen. Stellt man den Pumplaserstrahl so ein, dass er exakt 45° zu beiden Kristallen steht, werden statistisch gesehen gleich viele Photonenpaare in beiden Kristallen erzeugt. Dadurch geht ohne Messung die Information verloren, ob die Photonenpaare Horizontal oder Vertikal polarisiert sind.
Für den Durchgang der Photonen durch den Kristall gilt:
\begin{align*}
&\Ket{V}_{P}\rightarrow\Ket{H}_{s}\Ket{H}_{i}\\
&\Ket{H}_{P}\rightarrow\,e^{i\phi}\Ket{V}_{s}\Ket{V}_{i}
\end{align*}
Die Verschiebung der Phase um einen Winkel $\phi$ kann durch Zerstreuung innerhalb des Kristalls zustande kommen. Betrachtet man nun einen Pumplaserstrahl der Form
\begin{equation}
\Ket{\Psi}_{Pump}=cos(\theta)\cdot\Ket{V}_{P}+e^{i\phi}sin(\theta)\cdot\Ket{H}_{P}
\end{equation}
so schreib sich dieser nach dem Durchgang durch den Kristall als
\begin{equation}
\Ket{\Psi}_{DC}=cos(\theta)\cdot\Ket{H}_{s}\Ket{H}_{i}+e^{i2\phi}sin(\theta)\cdot\Ket{V}_{s}\Ket{V}_{i}
\end{equation}
Im Experiment wir nun $\theta$ mithilfe einer $\frac{\lambda}{2}$-Platte auf 45° eingestellt und $\phi$=0 angenommen. Damit erreicht man den gewollten Bell-Zustand \ref{bellzustand}.
\section{Das Einstein-Podolsky-Rosen Paradoxon}
Der Umstand, dass die Quantenmechanik verschränkte Zustände zulässt, die sich scheinbar mit beliebig hoher Geschwindigkeit über beliebige Distanzen beeinflussen, veranlasste A. Einstein, N. Rosen und B. Podolsky (EPR) 1935 dazu in einem Gedankenexperiment, dem EPR Paradox, aufzueigen, warum die Quantenmechanik durch eine lokale Theorie versteckter Variblen ergänzt/ersetzt werden müsste. Einstein sprach zunächst von einer spukhaften Fernwirkung "'spooky action at a distance"', da eine Messung an einem Teil eines verschränkten Systems einen Kollaps der Wellenfunktion des anderen Teilchens zur Folge hätte, dies jedoch eine Wechselwirkung mit der Umwelt benötigt.
Der Gegenpart und Verfechter der Quantenmechanik war Nils Bohr, welcher überzeugt von der nicht deterministischen Quantenmechanik war.
Für ihr Paradox verfassten EPR drei Grundlegende Annahmen physikalischer Theorien:
\begin{itemize}
\item\textbf{Vollständigkeit}: In einer vollständigen physikalischen Theorie hat jedes Element der physikalischen Realität einen Gegenpart in der Theorie.
\item\textbf{Realität}: Lässt sich der Wert einer physikalischen Größe mit einer Wahrscheinlichkeit von 1 ohne Messung am System vorhersagen, so gibt es ein dieser Größe entsprechendes Element in der physikalischen Realität.
\item\textbf{Lokalität}: Sind zwei Systeme aufgrund z.B. einer großen räumlichen Trennung nicht mehr Interaktionsfähig, so kann keine Änderung am zweiten System durch eine Störung (Messung) des ersten Systems bedingt sein.
\end{itemize}
Die Theorie der versteckten Variablen (engl.: hidden varaible theory - HVT) ist der Ansatz von EPR die Quantenmechanik durch eine lokale deterministische Theorie zu ersetzten. Als versteckte Variablen bezeichnet man eine Größe, die sich einer direkten Messung entzieht, aber die Messung andere Systemgrößen beeinflusst. In der HVT steht als das Messergebnis schon vor der Messung fest, auch wenn keine Kenntnis über die verborgene Variable vorliegt. Damit ist die HVT vor allem eine deterministische Theorie. Hinzu kommt noch, dass die HVT realistisch und lokal ist. Die scheinbar zufälligen Messergebnisse der Quantenmechanik versucht man in der HVT durch eine ungenaue Präparation des Ausgangszustandes zu erklären, die zur folge hat, dass auch in der HVT trotz Lokalität und Realität der Theorie verschiedene Ergebnisse zustande kommen können.
Um zu zeigen, dass die Quantenmechanik Systeme nicht korrekt beschreibt betrachten wir das Gedankenexperiment, das dem Paradox zugrunde liegt:
Betrachtet man zwei Teilchen (1,2), die bis zu einem gewissen Punkt wechselwirken und sich dann trennen und setzt man zum Beispiel den Gesamtimpuls als Null voraus, so lässt sich durch eine Messung des Impulses an Teilchen 1 aufgrund der Impulserhaltung sofort auf den Impuls des zweiten Teilchens schließen. Verfährt man ebenso mit einer Ortsmessung an Teilchen 2, lässt sich umgekehrt auf den Ort des ersten Teilchens schließen. Mit dieser Verfahrensweise lässt sich scheinbar die Heisenbergsche Unschärferelation $\Delta p\cdot\Delta x\geq\frac{\hbar}{2}$ verletzten. Dieses Problem wurde in der Koppenhagener Deutung der Quantenmechanik dadurch entkräftet, dass das Ermitteln des Impulses des zweiten Teilchens durch eine Messung am ersten keine Messung am zweiten Teilchen darstellt und dadurch kein Widerspruch zu Heisenberg zustande kommt.
\section{Bell'sche Ungleichung}
Bell formulierte 1964 eine alternative Version des EPR. Mit dieser Formulierung ist es möglich eine experimentelle Überprüfung des EPR durchzuführen. Die Formulierung von John F. Clauser, Michael A. Horne, Abner Shimony und Richard A. Holt (CHSH) stellt eine Verallgemeinerung der Bell'schen Ungleichung dar und lässt sich auf unser Experiment anwenden. Der sogenannte Bellwert
\[
S = E(a, b) - E(a, b^\prime) + E(a^\prime, b) + E(a^\prime, b^\prime)
\]
gibt einen greifbaren Wert für die Korrelation der gemessenen Eigenschaften an. Bei jeglicher klassischen, also realen und lokalen Theorie mit versteckter Variable liegt dieser Wert bei
\begin{equation}\label{eq:bellungl}
S\leq 2
\end{equation}
Verletzt man mit einem Experiment nun diese Ungleichung signifikant ist es nicht möglich dies mit einer realen lokalen Theorie mit versteckter Variable zu erklären und es muss Lokalität und/oder Realität aufgegeben werden.\\
In dem vorliegenden Experiment wird die Polarisation von verschränkten Photonen betrachtet. $ a,a',b $ und $ b' $ sind also vier verschiedene Polarisationswinkel. Der Erwartungswert $ E(a,b) $ berechnet sich über die Wahrscheinlichkeiten unter den Winkeln $ a $ und $ b $ polarisierte Photonen koinzident zu messen.
\begin{align*}
E(a, b)=p_{VV}(a,b)+p_{HH}(a,b)-p_{HV}(a,b)-p_{VH}(a,b)
\end{align*}
wobei $p_{VV}(a,b)$ die Wahrscheinlichkeit ist, dass A und B unter den Winkeln a und b beide V messen. Um die \glqq nicht durchgegangenen\grqq\;Ereignisse $N_{HH}(a,b)$ zu messen drehen wir a und b um 90\degree\;und machen quasi H zu V, damit diese wieder durchgehen. $p_{VV}(a,b)$ ist also
\begin{align*}
p_{VV}(a,b) &= \frac{N_{VV}(a,b)}{N_{VV}(a,b)+N_{HH}(a,b)+N_{VH}(a,b)+N_{HV}(a,b)}\\
&=\frac{N(a,b)}{N(a,b)+N(a+90\degree,b+90\degree)+N(a,b+90\degree)+N_{HV}(a+90\degree,b)}
\end{align*}
Es müssen also für jeden Erwartungswert vier Zählraten gemessen werden. Für den Bellwert folglich mindestens 16.
\section{Vergleich Quantenmechanik und klassische Theorie}
Wie bereits erwähnt liegt der Bellwert jeder klassischen Theorie bei maximal 2 (\ref{eq:bellungl}).\\
Quantenmechanisch betrachtet verändert ein Polarisator unter dem Winkel $a$ den Zustand $ \Ket{V_a} $ zu
\begin{equation}\label{eq:pol}
\Ket{V_a}=\cos(a)\Ket{V}-\sin(a)\Ket{H}
\end{equation}
und $\Ket{H_a}$ zu 
\begin{equation}
\Ket{H_a}=\sin(a)\Ket{V}+\cos(a)\Ket{H}
\end{equation}
Damit ist die Wahrscheinlichkeit einer Koinzidenzmessung
\begin{align*}
p_{VV}(a,b) &= |\Bra{V_a}\Bra{V_b}\Ket{\Psi_{\text{Bell}}}|^2\\
&=\frac{1}{2}|\Bra{V_a}_s\Bra{V_b}_i(\Ket{H}_s\Ket{H}_i+\Ket{V}_s\Ket{V}_i)|^2\\
&=\frac{1}{2}|\Bra{V_a}_s\Bra{V_b}_i\Ket{H}_s\Ket{H}_i+\Bra{V_a}_s\Bra{V_b}_i\Ket{V}_s\Ket{V}_i|^2\\
&=\frac{1}{2}|\Bra{V_a}_s\Ket{H}_s\Bra{V_b}_i\Ket{H}_i+\Bra{V_a}_s\Ket{V}_s\Bra{V_b}_i\Ket{V}_i|^2\\
&\overset{\ref{eq:pol}}{=}\frac{1}{2}|-\sin(a)\cdot -\sin(b)+\cos(a)\cos(b)|^2\\
&=\frac{1}{2}|\cos(a-b)|^2
\end{align*}
Hier ist die Abhängigkeit der Koinzidenzwahrscheinlichkeit, im Gegensatz zur klassischen Darstellung, proportional zu $ \cos^2 $ und nicht linear.\\
Berechnet man aus dieser und den anderen Wahrscheinlichkeiten die Erwartungswerte sagt die Quantenmechanik für das Tupel $ (a,a',b,b')=(0\degree,45\degree,22,5\degree,67,5\degree) $, und alle mit gleichen Eigenschaften, einen maximalen Bellwert von $ S=2\sqrt{2} $ voraus, was eine signifikante Abweichung vom klassischen Wert darstellt
\section{Halbleiter}
In diesem Versuch werden Avalanche-Photodioden (ADP) zur Detektion der Photonen und ein Halbleiterlaser benutzt. Beide basieren auf der Halbleitertechnik.
\subsection{Avalanche-Photodiode}\label{diode}
Avalanche-Photodioden erzeugen, ähnlich wie bei einem Photomultiplier, durch den Photoeffekt ein primäres Elektron, welches dann lawinenartig (daher der Name) verstärkt wird. Das Elektron wird aufgrund des hohen Feldes stark genug beschleunigt, sodass es weitere Elektronen herauslöst, welche wiederum beschleunigt werden (siehe Abb. \ref{fig:apd}). Dies führt zu einem messbaren Strom. Betrieben werden APD bei Sperrspannungen nahe der Durchschlagsspannung. Während eines Ereignisses wird die Spannung heruntergefahren um einen Durchschlag und somit eine Beschädigung der Diode zu verhindern. 
\subsection{Halbleiterlaser}\label{laser}
Die Laserdiode weist alle Merkmale eines Lasers auf. Die angeschlossene Spannungsquelle fungiert als Pumpe, die verspiegelten Flächen als Resonator und der p-n-Übergang als Medium.\\
Wie bei einer LED entstehen die Photonen durch die Rekombination von Elektronen und Löchern, wobei das Elektron auf einem energetisch höheren Niveau ist als das Loch und Energiedifferenz als Photon emittiert. Ist der Pumpstrom hoch genug, kommt es zu einer Besetzungsinversion und es tritt stimulierte Emission auf.\\
Bei dem in Abb. \ref{fig:Laser} dargestellten tritt das Laserlicht an den Seiten aus. Es handelt sich um einen Kantenemitter.
\begin{figure}[H]
  \centering
    \subfloat[Ladungslawine in einer APD]{\label{fig:apd}
    \includegraphics[width=0.45\textwidth]{Graphics/APD.pdf}}\quad 
  \subfloat[Schematischer Aufbau einer Laserdiode (Kantenemitter)]{\label{fig:Laser}
    \includegraphics[width=0.45\textwidth]{Graphics/Diodenlaser.pdf}}\quad   
  \caption{Halbleiterbauteile}
  \label{fig:halbleiter}
\end{figure}
\chapter{Versuchsdurchführung und Auswertung}
\section{Aufbau}
\begin{figure}[H]
\centering
   	\begin{minipage}[b]{1.0\textwidth}
   	\includegraphics[width=\textwidth]{Graphics/aufbau.pdf}
   	\caption{Messreihe mit Fitfunktion}
  	\label{3dfit}
   	\end{minipage}
\end{figure}
Um einen Überblick über den Versuchsaufbau zu erhalten, werden nun die wichtigsten Komponenten erläutert:
\begin{itemize}
\item[Laser]: Als Pumplaser wird ein $405\,\si{nm}$ Diodenlaser verwendet, dessen genauerer Aufbau in \ref{laser} erklärt wurde. Über zwei Spiegel, die bereits vor dem Versuch eingestellt waren und damit nicht mehr justiert werden mussten, wird der Strahl umgelenkt und so auf den weiteren Aufbau ausgerichtet.
\item[1]: Um die Intensität $I\propto\,E^2$ zu erhöhen, wird der Strahl durch ein Teleskop aus einer Konvex- und einer Konkavlinse auf einen Durchmesser von ca. $1\,\si{mm}$ kollimiert.
\item[2]: Eine $\frac{\lambda}{2}-Platte$ sorgt dafür, dass der Strahl an Pumpphotonen mit einer Polarisation von 45° zu den Flächennormalen der beiden Kristalle steht.
\item[3]: Der BBO-Kristall (Barium-Borat) bzw. der Verbund der beiden Kristalle dient als Quelle der polarisationsverschränkten Photonen. Diese entstehen über den Prozess der parametrischen Konversion, die bereits in \ref{epz} erläutert wurde. Verwendet wird ein Typ I Kristall, so dass die verschränkten Photonen den Kristall diametral auf einem Konus verlassen.
\item[4]: Ein Strahlblocker sorgt dafür, dass die Photonen, die nicht in der SPDC umgewandelt wurden, nicht in den empfindlichen Detektor gelangen.
\item[5]: Die Polarisatoren werden während des Versuches gedreht, um so die verschiedenen Winkel einzustellen, die in der Vorbereitung angesprochen wurden. Beide sind symmetrisch im Strahlengang eingebaut.
\item[6]: Eine Irisblende vor dem Detektor sorgt dafür, dass weniger Streulicht in diesen gelangt und bündelt den Strahl schon grob.
\item[7]: Eine weitere Sammellinse fokussiert den Strahl auf die Detektorfläche der beiden Detektoren.
\item[8]: Alice und Bob,die beiden Detektoren sind Einzel-Photonen-Zählmodule aus einer Avalanche-Photodiode \ref{diode}, die sogar einzelne Photonen detektieren können. Aufgrund dieser hohen Empfindlichkeit wurde der Versuch in einen bis auf einen PC-Bildschirm abgedunkelten Raum durchgeführt. 
\end{itemize}
\section{Vorbereitende Messungen}
\subsection{Laserschwelle}
Zunächst wurde eine Strom-Leistungs-Kennlinie des $\lambda=405\,\si{nm}$ Halbleiterlasers aufgenommen, um dessen Laserschwelle zu bestimmen. Aufgenommen wurden Messwerte im Abstand von $5\,\si{mA}$ bzw. im vermuteten Bereich der Laserschwelle im Anstand von $1\,\si{mA}$. Anschließend wurde eine Ausgleichsgerade der Form
\begin{equation}
P_{Laser}(I)=m\cdot\left(I+I_{0}\right)
\end{equation}
mittels Python an die Daten angefittet. Der Fit ergab mit den von Python ausgegebenen Standartabweichungen eine Laserschwelle von $I_{0}=\left(38,985\pm\,0,023\right)\,\si{mA}$ und eine Steigung von $m=\left(1,398\pm\,0,002\right)\,\si{\frac{mW}{mA}}$.
\begin{figure}[H]
\centering
   	\begin{minipage}[b]{1.0\textwidth}
   	\includegraphics[width=\textwidth]{Graphics/Laserschwelle.pdf}
   	\caption{Messreihe mit Fitgerade}
  	\label{laserschwelle}
   	\end{minipage}
\end{figure}
Die Fehlerbalken der Messwerte lassen sich aufgrund ihrer geringen Größe nicht erkennen.
\subsection{$\frac{\lambda}{2}$-Platte}
Wie bereits in \ref{epz} erläutert soll der Pumplaserstrahl eine Polarisation von 45° gegenüber beiden Flächennormalen der BBo Kristalle aufweisen. Hierfür muss eine $\frac{\lambda}{2}$-Platte im Strahlengang jusiert werden. Es werden unter den Winkeleinstellungen der Polarisatoren $\alpha\,=\beta\in$[0°,180°] mit $\Delta\alpha\,=20°$ Messungen der Intensität in Abhängigkeit der Stellung der $\frac{\lambda}{2}$-Platte durchgeführt. Die sich ergebenden Graphen sind in \ref{bellprep} und \ref{bellprerp1} zu sehen. Anhand des ersten Schnittpunktes der Kurvenschar wurde der optimale Winkel der $\frac{\lambda}{2}$-Platte zu $\phi\,=7,8°\pm0,5°$ bestimmt werden. Dieser Winkel bleibt für den restlichen Verlauf des Experimentes unverändert.
\begin{figure}[H]
  \centering
  \subfloat[Plot der gesamten Messreihe]{\label{bellprep}
    \includegraphics[width=0.45\textwidth]
    {Graphics/bell_prep_1.pdf}}\quad
  \subfloat[Ausschnitt zur Bestimmung des Schnittpunktes ]{\label{bellprerp1}
    \includegraphics[width=0.45\textwidth]
    {Graphics/bell_prep_1.pdf}}\quad
  \caption{ }
  \label{resisspule}
\end{figure}
\section{Messungen zur Bestimmung des Bellwertes}
\subsection{Polarisationskorrelaation}
Für die erste eigentliche Messreihe zur Bestimmung des Bellwertes wurde einer der beiden Polarisatoren auf einen festen Winkel $\alpha=0°,45°,90°,135°$ eingestellt und zu jedem dieser Winkel eine Messreihe über $\beta\in[0°,360°]$ mit $\Delta\beta=20°$ mit zwei Koinzidenzmessungen über je $t=10\,\si{s}$ aufgenommen. Da keiner der einzelnen Zählraten von der zweiten auffällig abweicht, wird für den weiteren Verlauf und den anschließenden Fit der Mittelwert der Zählraten und der dazugehörigen zufälligen Koinzidenzen verwendet.
Um für den Fit nur die wahren Koinzidenzen zu betrachten, müssen die zufälligen von der Gesamtheit abgezogen werden, sodass sich mit den gesamten $N_{1,2g}$ und den zufälligen Koinzidenzen $N_{1,2z}$ ergibt $N_{wahr}=\frac{N_{1g}+N_{2g}-2N_{1z}-2N_{2z}}{2}$ sowie für den Fehler $N_{wahr}=\frac{\sqrt{N_{1g}+N_{2g}+4N_{1z}+4N_{2z}}}{2}$. Dabei wurden die zufälligen Koinzidenzen aufgrund der Eigenart der Messelektronik doppelt subtrahiert. An die Daten wurde mittels Python ein 3D-Fit folgender Funktion angefertigt:
\begin{equation}
N(\alpha,\beta)=A\cdot\cos^2(\alpha-\beta+B)+C
\end{equation}
Der Fit zusammen mit den Messwerten ist in \ref{3dfit} zu sehen:
\begin{figure}[H]
\centering
   	\begin{minipage}[b]{1.0\textwidth}
   	\includegraphics[width=\textwidth]{Graphics/3Dplot_mit_fit.pdf}
   	\caption{Messreihe mit Fitfunktion}
  	\label{3dfit}
   	\end{minipage}
\end{figure}
Der Fit wurde sowohl an die Daten und die Fehler der einzelnen Messpunkte, sowie an die Unsicherheit der Variablen $\Delta\alpha,\beta=\pm1°$ angefertigt und ergab für die einzelnen Parameter A,B,C folgende Werte, wobei als Fehler der durch das Fitprogramm ausgegebene Standardfehler angegeben wurde:
\begin{align}
A&=3033,92\pm\,18,28\,\si{Counts\,pro\,10s} \\
B&=0,144\pm\,0,005\,\si{rad} \\
C&=39,95\pm\,8,17\,\si{Counts\,pro\,10s}
\end{align}
Um das Maß der Photonenverschränkung zu ermitteln, wird zunächst der Michelson-Kontrast 
\begin{equation}
K_{Michelson}=\frac{N_{max}-N_{min}}{N_{max}+N_{min}}=\frac{A}{A+2C}
\end{equation}
betrachtet. Dieser berechnet sich über die Parameter aus dem Fit zu $K_{Michelson}=0,974\pm0,005$. Ein Michelsonkontrast von $K_{Michelson}=1$ steht hierbei für 100\% verschränkte Photonen und $K_{Michelson}=0$ für 0\% Verschränkung der Photonen. Der Wert liegt nahe bei 1 und zeigt, dass nur ein kleiner Teil der Photonen nicht verschränkt ist.
\subsection{Bestimmung aus Fit}
testtest
\subsection{Bestimmung durch direkte Messung}
Um einen Bellwert aus einer direkten Messung zu erhalten, wurden die Bellwerte direkt eingestellt und die erforderlichen Zählraten gemessen. Jede nötige Winkelkombination wurde zweimal über eine Dauer von t=10s gemessen. Weicht keiner der Werte signifikant vom anderen ab, bildet der Mittelwert der Messwerte die Grundlage der Weiteren Berechnungen. Tabelle \ref{bellwerte} zeigt die Messwerte
\begin{table}[H]\label{bellwerte}
\renewcommand*{\arraystretch}{1.2}
\centering
\begin{tabular}{|c|c|c|c|c|}
\hline 
$\beta(\downarrow)\alpha(\rightarrow)$in ° & 0 & 45 & 90 & 135 \\ 
\hline 
22,5 & 2195 & 2820 & 879,5 & 181,5 \\ 
\hline 
67,5 & 247,5 & 2278,5 & 2961 & 920 \\ 
\hline 
112,5 & 869 & 265 & 2377 & 2897,5 \\ 
\hline 
157,5 & 2762,5 & 728,5 & 219,5 & 2234 \\ 
\hline 
\end{tabular} 
\caption{Messwerte aus Messung mit direkten Bellwinkeln}
\end{table}
Aus diesen Daten berechnet sich der Bellwert über 
\begin{equation}
S=E(0°,22.5°)-E(0°,67.5°)+E(45°,22.5°)+E(45°,67.5°)=2,6158
\end{equation}
Der Fehler $\Delta$S berechnet sich über
\begin{equation}
\Delta\,S=\sqrt{\sum_{E}\Delta\,E}
\end{equation}
wobei sich jedes einzelne $\Delta$E berechnet über
\begin{equation}
\Delta\,E(\alpha\,,\beta)=\sqrt{\sum_{i=1}^4\left(\frac{\delta\,E}{\delta\,N_{i}}\right)\cdot\,N_{i}}
\end{equation}
Damit berechnet sich der Bellwert zu $S=2,6158\pm\,0,0162$. Dieser Wert liegt zwar auch in der oberen Grenze noch um ca. 7\% von dem nach der Quantenmechanik zu erwartenden Wert von $S=2\sqrt{2}\approx\,2,8284$ jedoch liegt er etwas über das 38 Fache der Standartabweichung über dem nach einer HVT prognostizierten Wert S=2.
\chapter{Fazit}	
•
\renewcommand{\bibname}{Literatur}
\begin{thebibliography}{0}
\bibitem {refname} 
•
\end{thebibliography}
\end{document}    
